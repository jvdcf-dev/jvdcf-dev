\documentclass[10pt, letterpaper]{article}

% Packages:
\usepackage[
    ignoreheadfoot, % set margins without considering header and footer
    top=2 cm, % seperation between body and page edge from the top
    bottom=2 cm, % seperation between body and page edge from the bottom
    left=2 cm, % seperation between body and page edge from the left
    right=2 cm, % seperation between body and page edge from the right
    footskip=1.0 cm, % seperation between body and footer
    % showframe % for debugging 
]{geometry} % for adjusting page geometry
\usepackage{titlesec} % for customizing section titles
\usepackage{tabularx} % for making tables with fixed width columns
\usepackage{array} % tabularx requires this
\usepackage[dvipsnames]{xcolor} % for coloring text
\definecolor{primaryColor}{RGB}{0, 79, 144} % define primary color
\usepackage{enumitem} % for customizing lists
\usepackage{fontawesome5} % for using icons
\usepackage{amsmath} % for math
\usepackage[
    pdftitle={@jvdcf - Software Engineer Student},
    pdfauthor={João Vítor da Costa Ferreira},
    colorlinks=true,
    urlcolor=primaryColor
]{hyperref} % for links, metadata and bookmarks
\usepackage[pscoord]{eso-pic} % for floating text on the page
\usepackage{calc} % for calculating lengths
\usepackage{bookmark} % for bookmarks
\usepackage{lastpage} % for getting the total number of pages
\usepackage{changepage} % for one column entries (adjustwidth environment)
\usepackage{paracol} % for two and three column entries
\usepackage{ifthen} % for conditional statements
\usepackage{needspace} % for avoiding page brake right after the section title
\usepackage{iftex} % check if engine is pdflatex, xetex or luatex

% Ensure that generate pdf is machine readable/ATS parsable:
\ifPDFTeX
    \input{glyphtounicode}
    \pdfgentounicode=1
    % \usepackage[T1]{fontenc} % this breaks sb2nov
    \usepackage[utf8]{inputenc}
    \usepackage{lmodern}
\fi



% Some settings:
\AtBeginEnvironment{adjustwidth}{\partopsep0pt} % remove space before adjustwidth environment
\pagestyle{empty} % no header or footer
\setcounter{secnumdepth}{0} % no section numbering
\setlength{\parindent}{0pt} % no indentation
\setlength{\topskip}{0pt} % no top skip
\setlength{\columnsep}{0cm} % set column seperation
\makeatletter
\let\ps@customFooterStyle\ps@plain % Copy the plain style to customFooterStyle
\patchcmd{\ps@customFooterStyle}{\thepage}{
    % \color{gray}\textit{\small Template: {\href{https://www.overleaf.com/latex/templates/rendercv-sb2nov-theme/gdspgtsnfncm}{RenderCV}}} 
}{}{} % replace number by desired string
\makeatother
\pagestyle{customFooterStyle}

\titleformat{\section}{\needspace{4\baselineskip}\bfseries\large}{}{0pt}{}[\vspace{1pt}\titlerule]

\titlespacing{\section}{
    % left space:
    -1pt
}{
    % top space:
    0.3 cm
}{
    % bottom space:
    0.2 cm
} % section title spacing

\newenvironment{highlights}{
    \begin{itemize}[
        topsep=0.10 cm,
        parsep=0.10 cm,
        partopsep=0pt,
        itemsep=0pt,
        leftmargin=0.4 cm + 10pt
    ]
}{
    \end{itemize}
} % new environment for highlights

\newenvironment{highlightsforbulletentries}{
    \begin{itemize}[
        topsep=0.10 cm,
        parsep=0.10 cm,
        partopsep=0pt,
        itemsep=0pt,
        leftmargin=10pt
    ]
}{
    \end{itemize}
} % new environment for highlights for bullet entries


\newenvironment{onecolentry}{
    \begin{adjustwidth}{
        0.2 cm + 0.00001 cm
    }{
        0.2 cm + 0.00001 cm
    }
}{
    \end{adjustwidth}
} % new environment for one column entries

\newenvironment{twocolentry}[2][]{
    \onecolentry
    \def\secondColumn{#2}
    \setcolumnwidth{\fill, 6.4 cm}
    \begin{paracol}{2}
}{
    \switchcolumn \raggedleft \secondColumn
    \end{paracol}
    \endonecolentry
} % new environment for two column entries

\newenvironment{header}{
    \setlength{\topsep}{0pt}\par\kern\topsep\centering\linespread{1.5}
}{
    \par\kern\topsep
} % new environment for the header

\newcommand{\placelastupdatedtext}{% \placetextbox{<horizontal pos>}{<vertical pos>}{<stuff>}
  \AddToShipoutPictureFG*{% Add <stuff> to current page foreground
    \put(
        \LenToUnit{\paperwidth-2 cm-0.2 cm+0.05cm},
        \LenToUnit{\paperheight-1.0 cm}
    ){\vtop{{\null}\makebox[0pt][c]{
        \small\color{gray}\textit{Last updated in January 2025}\hspace{\widthof{Last updated in September 2024}}
    }}}%
  }%
}%

% save the original href command in a new command:
\let\hrefWithoutArrow\href

% new command for external links:
\renewcommand{\href}[2]{\hrefWithoutArrow{#1}{\ifthenelse{\equal{#2}{}}{ }{#2 }\raisebox{.15ex}{\footnotesize \faExternalLink*}}}


\begin{document}
    \newcommand{\AND}{\unskip
        \cleaders\copy\ANDbox\hskip\wd\ANDbox
        \ignorespaces
    }
    \newsavebox\ANDbox
    \sbox\ANDbox{}

    \placelastupdatedtext
    \begin{header}
        \begin{twocolentry}
            {\normalsize
            \mbox{{\color{black}\footnotesize\faMapMarker*}\hspace*{0.13cm}Santo Tirso, Portugal}%
            \kern 0.25 cm%
            \AND%
            \kern 0.25 cm%
            \mbox{\hrefWithoutArrow{https://linkedin.com/in/jvdcf}{\color{black}{\footnotesize\faLinkedinIn}\hspace*{0.13cm}@jvdcf}}%
            \kern 0.25 cm%
            \AND%
            \kern 0.25 cm%
            \mbox{\hrefWithoutArrow{https://github.com/jvdcf}{\color{black}{\footnotesize\faGithub}\hspace*{0.13cm}@jvdcf}}
            \kern 0.25 cm%
            \AND%
            \kern 0.25 cm%
            \mbox{\hrefWithoutArrow{mailto:jvdcf+jobs@proton.me}{\color{black}{\footnotesize\faEnvelope[regular]}\hspace*{0.13cm}jvdcf+jobs@proton.me}}%
            }
            {\textbf{\fontsize{24 pt}{24 pt}\selectfont João Ferreira}
            
            \vspace{0.2 cm}
    
            \textit{3rd year student of L.EIC @ FEUP}
            }
            
        \end{twocolentry}
        
    \end{header}

    \vspace{0.3 cm}

% =============================================================================

    As a dedicated Software Engineer student, I am passionate about building near-perfect solutions with \textbf{clean and bug-free code}, \textbf{reproducible deployment}, and everything in between.
    I strive to deliver products with \textbf{explicit} documentation and intentions, always with the user in mind, while keeping an \textbf{adventurous mindset} to learn.

    \vspace{0.3 cm}

    \section{Education}
        \begin{twocolentry} {
                \textit{Sept 2022 - July 2025}

                \textit{\href{https://github.com/jvdcf/feup}{feup.git}}
            } {            
                \textbf{Bsc in Informatics and Computing Engineering}

                \textit{Faculty of Engineering, University of Porto, Portugal}
            }
        \end{twocolentry}
        \vspace{0.10 cm}
        \begin{onecolentry}
            Broad education about programming, algorithms, computer 
            architecture, operating systems, computer networks, software 
            engineering, databases, cybersecurity, computer graphics, 
            interaction and AI.
        \end{onecolentry}
    \vspace{0.3 cm}

    \section{Main Projects (see more in the LinkedIn profile)}
        
        \begin{twocolentry}
            {\textit{Dec 2024}}
            {\textbf{Deployment and study of a network}}
        \end{twocolentry}
        \vspace{0.1 cm}
        \begin{onecolentry}
            With a MikroTik router and switch and 4 Linux machines, an internal network capable of downloading a file from an FTP server was deployed with bridges, NAT, and manual routes.
            This project includes an FTP client built in C using sockets and a report detailing every step of the process and explaining every phenomenon.
        \end{onecolentry}

        \vspace{0.2 cm}

        \begin{twocolentry}
            {\textit{\href{https://github.com/jvdcf/da-salesperson}{da-salesperson.git}}}
            {\textbf{Routing algorithms in a delivery context}}
        \end{twocolentry}
        \vspace{0.1 cm}
        \begin{onecolentry}
            A C++ TUI application that implements algorithms and heuristics to solve the Traveling Salesperson Problem in various ways, with code documentation in Doxygen.
            Despite not being focus of the project, the repository also includes GitHub Actions used in the development process to ensure the code was valid in GCC and Clang.
        \end{onecolentry}

        \vspace{0.2 cm}

        \begin{twocolentry}
            {\textit{Since 2021}}
            {\textbf{Deployment of a smart home with Home Assistant}}
        \end{twocolentry}
        \vspace{0.1 cm}
        \begin{onecolentry}
            In a Raspberry Pi 4, Home Assistant was deployed in a home, in conjunction with Shelly smart energy meters, switches, a solar and battery inverter, and a car charger, to make the most use of the solar production, more savings and remote control of the house. A great project which involves IoT, home automation, SSL certificates, HTTPS, ModBus TCP, network security, critical infrastructure, Python, CSS, and much more.
        \end{onecolentry}

        \vspace{0.2 cm}

        \begin{twocolentry}
            {\textit{\href{https://github.com/jvdcf-dev/nix}{nix.git}}}
            {\textbf{Declarative Linux configuration in Nix}}
        \end{twocolentry}
        \vspace{0.1 cm}
        \begin{onecolentry}
            My personal NixOS configuration for software development, where "it works on my machine" is not a problem anymore and with a focus on reproducibility and compatibility.
        \end{onecolentry}

    \vspace{0.3 cm}
    \section{Tecnologies}

    \begin{twocolentry}
        {\textit{Programming Languages}}
        {C, C++, Bash, Python, PHP, Laravel, SQLite, PostgreSQL, Rust, Nix, Java, HTML, CSS}
    \end{twocolentry}
    \vspace{0.1 cm}
    \begin{twocolentry}
        {\textit{Tools}}
        {Linux, Git, Docker, GitHub Actions, Doxygen, GitHub/GitLab}
    \end{twocolentry}
    
    \vspace{0.3 cm}
    \section{Volunteering Work}

    \begin{twocolentry}
        {\textit{\href{https://eic30anos.fe.up.pt/pt/event/}{eic30anos.fe.up.pt}}}
        {\textbf{Volunteer in "30 anos da EIC"}}
    \end{twocolentry}
    \vspace{0.1 cm}
    \begin{onecolentry}
        In the celebration of the 30 years of the EIC course, I was responsible for guiding the participants in the event and ensuring an overall good experience.
    \end{onecolentry}

    \vspace{0.2 cm}

    \begin{twocolentry}
        {\textit{\href{https://www.studocu.com/pt/user/joao-vitor-ferreira/9070933}{studocu.com}}}
        {\textbf{Notes and summaries for Portuguese students}}
    \end{twocolentry}
    \vspace{0.1 cm}
    \begin{onecolentry}
        All the notes I made during high school (Technologies and Sciences course) were freely shared with more than 21 thousand students, helping them succeed in their national exams.
    \end{onecolentry}

    \section{Languages}
    Portuguese (Native) and English (Intermediate).

\end{document}
